\documentclass[a4paper]{article}

\usepackage[english]{babel}
\usepackage[utf8x]{inputenc}
\usepackage{amsmath}
\usepackage{graphicx}
\usepackage[colorinlistoftodos]{todonotes}
\usepackage{hyperref}
\usepackage{listings}

\title{tp \- linux in a nutshell}
\author{Thomas Michelot \- Raphael Gault \- Fabien Siron \- Francois Mazeau}

\begin{document}
\lstset{language=bash}
\maketitle
\newpage
\tableofcontents
\newpage
\section{Where is the pdf??}
Dans cette section, nous allons voire comment obtenir le pdf du tp dans
sa machine.
\subsection{find a terminal}
Avant tout, ouvrons un terminal\ldots 
\begin{itemize}
  \item Avec i3, il suffit d'appuyer sur mod et enter simultanement.
  \item Sur Ubuntu, avec Unity, il suffit d'utiliser Ctrl + Maj + t  %TODO -> find graph'manager for the epita ubuntu based distro
\end{itemize}
\subsection{clone me}
Nous voulons maintenant cloner le repo git du tp dans notre ordinateur (si vous
ne savez pas ce qu'est git, % panic a bord, on fait quoi?
).
Il suffit d'entrer:
\begin{lstlisting}
> git clone https://bitbucket.org/gconfs/linux-tp
\end{lstlisting}

\subsection{find the tp directory}
Maintenant, nous allons apprendre nos deux premieres commandes: ls et cd.
\begin{itemize}
  \item \emph{ls} (LiSt) permet de lister les fichiers et dossiers du repertoire
    courant.
  \item \emph{cd} (Change Directory) permet de changer de repertoire courant.
    \begin{itemize}
      \item \emph{cd 42} va dans 42/ si il existe dans le repertoire courant et que
        c'est un repertoire.
      \item \emph{cd . } ne fait rien. (en fait, . represente le repertoire
        courant).
      \item \emph{cd ..} remonte dans l'arborescence. (user/who/42,user/who)
      \item \emph{cd -} reviens au dernier repertoire courant.
    \end{itemize}
\end{itemize}
Il nous faut donc aller dans le repertoire du tp\ldots
\subsection{make me}
Verifiez qu'il y a bien un Makefile (voire ci-dessus) et ensuite rien de plus
simple:
\begin{lstlisting}
> make
\end{lstlisting}
\subsection{read me}
Pour lire le tp, verifiez qu'il y est bien un pdf dans le repertoire d'abord,
ensuite:
\begin{lstlisting}
> evince tp.pdf
\end{lstlisting}
Evince est un simple lecteur de pdf, nous verrons plus tard comment acceder aux
programmes.
\newpage


\section{Let's play with the command line}
\subsection{Let's get the files}
\paragraph{}Pour cet exercice vous allez devoir télécharger l'archive suivante,\\
http://sousdomaine.domaine.fr/directory/file.tar.gz \\ 
bien entendu, il n'est pas interessant de le faire via le navigateur.\\
Renseignez vous donc sur \emph{curl} et \emph{wget}. Si ils ne sont pas sur 
votre systeme, cela fera un bon execice de familiarisation avec votre gestionnaire 
de paquets. (apt-get pour ubuntu, pacman pour arch).

\subsection{Extraction}
\paragraph{}Votre prochaine mission est d'extraire les fichier via la ligne de 
commande. LISEZ LE MANUEL. Indice, commande file.

\subsection{Let's get started}
\paragraph{}Vous l'avez surement déjà fait mais on ne sait jamais : listez les 
fichiers extraits.\\
Vous constaterez qu'il y en a un certain nombre, on va donc les mannipuler!\\
Commencez par supprimer tous les fichiers à l'extension \emph{.txt}.\\
\paragraph{}Maintenant, supprimez tous les fichiers qui commences par une 
lettre \emph{entre a et e}.
\paragraph{}Ensuite, créez des fichiers appelés test suivis d'un nombre entre 
1 et 100 (ex : test42).
\paragraph{}Dernière étape, supprimez tous les fichiers tests suivis de 2 ou 
3 chiffres.


\section{Edit the environment}

\section{Text Editor}
Les tutoriels integres aux editeurs sont parmi les meilleurs, alors pourquoi
s'en priver?
\subsection{Vi/Vim} % the best
\begin{lstlisting}
> vimtutor
\end{lstlisting}
\subsection{Emacs}
\begin{lstlisting}
> emacs -nw # in console
# Ctrl-h t
\end{lstlisting}

\section{Play with processus}
Un processus est une instance d'un programme. Quand vous allumez deux terminaux,
vous creez deux processus, meme si c'est le meme programme. Dans linux, les
processus sont organises sous la forme d'un arbre, chaque processus peut en
creer d'autres et tuer un processus tue aussi ses fils.
\subsection{Watch the processus}
Differentes possibilites:
\subsubsection{ps}
\begin{lstlisting}
> ps # affiche les processus fils d'une console
> ps -ax # affiche tous les processus
> pstree # affiche un arbre des processus
\end{lstlisting}
\subsubsection{top/htop}
\begin{lstlisting}
> top
> htop
\end{lstlisting}
\section{Advanced part}

\subsection{Your own script}
Vous allez maintenant créer votre premier script. Pour ce faire, vous aurez 
besoin de 2 commandes : lscpu et lspci !\\
Ces commandes vous fournissent des informations sur votre micro-processeur et 
vos periphériques.\\
Le but du script est d'afficher le nom de votre cpu et de votre carte audio 
ainsi que votre carte réseau (network controller).\\
Hint : 
\begin{lstlisting}
> grep 
\end{lstlisting}
Hint 2 : 
\begin{lstlisting}
> #!/bin/bash
\end{lstlisting}
Hint 3 :
> chmod (rendre executable)
\\
Vous pourrez ajouter votre script à la variable PATH\\
help: \\
\url{http://steve-parker.org/sh/cheatsheet.pdf}

\subsection{monitoring}
Vous pouvez aussi vous amuser à parser /proc/stat, et à les afficher
en temps réel avec par exemple un rafraichissement de 1 seconde.
Vous aurez besoin de clear, de sleep et des expressions rationnelles.

\section{Some challenges}
Voici quelques challenges que vous pouvez relever avec ce tp et pas mal de
curiosite:



\url{http://ctf.lse.epita.fr/e/s/misc50/}

\url{http://ctf.lse.epita.fr/e/s/misc50-1/}

\url{http://ctf.lse.epita.fr/e/s/misc50-2/} \
\

rappel: le but n'est pas de faire le tp, mais de l'ouvrir \dots


\section{Home made challenges}

Pour ouvrir le pdf de ce challenge vous devrez vous servir de quelques commandes de base.\\


\url{http://dotty.fr/tp42.tgz}

\end{document}
